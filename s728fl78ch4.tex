Early-life exposure to factors associated with poverty is shown to have
adverse long-term effects on labor market outcomes, educational
attainment, and mortality~\citep{Almond_2011}. Hence, many welfare
programs are established to alleviate these effects and help children.
Evidence suggests that parental income is one of the strongest
predictors of one's educational attainment~\citep{Barrow_2012} and
health~\citep{Case_2002}. However, the question of whether cash
transfers to poor families impact children's longevity, educational
attainment, income,~and health is not yet answered.\\
The difficulty in assessing programs as such arises from the many
reasons why cash transfers could potentially fail to help children.
Furthermore, from an evaluation point of view identifying a plausible
counterfactual (i.e.~the outcome in the absence of the treatment) is as
difficult as obtaining data on long-term outcomes. The authors manage to
tackle these difficulties by collecting administrative records from the
Mothers' Pension (MP) program (1911-1935) - the first US
government-sponsored program targeting mothers with dependent children.
As individuals were not eligible for other programs at the time, the
analysis manages to isolate the effect of the unconditional cash
transfers.\\
To investigate the long-term outcomes, the authors track longevity, educational
attainment, health and labor market outcomes of the children of accepted
and rejected applicants. Using the children of mothers who applied and
were deemed eligible but later rejected as a counterfactual, the effect
on longevity was estimated to be approximately a 1-year increase for the
children of accepted mothers. This effect is greater for the poorest
families and very robust to alternative functional form specifications,
alternative counterfactual comparisons, and various treatment of attrition.
Furthermore, to examine the mechanisms through which longevity
increases, different channels are investigated through matching the MP
data with WWII enlistment and 1940 census records. Results yield that
cash transfers reduced the probability of being underweight, increased
educational attainment by 0.34 years, and increased income by 14\%.\\
The analysis excludes females as they are harder to follow, given that
they often change their names after marriage and due to their bad
representation, African-Americans are also not studied. Moreover, the
study suffers from attrition, which is tackled and does not bias the
results.\\
In conclusion, the authors claim that these unconditional cash transfers
manage to improve early life conditions enough to result in better medium- and
long-term outcomes.