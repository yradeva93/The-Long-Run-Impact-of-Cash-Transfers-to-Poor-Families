\textbf{Strengths and Contribution:} A. Aizer, S. Eli, J. Ferrie, and A. Lleras-Muney present the first study that identifies positive effects of cash transfers to poor mothers on childrens' longevity. The accomplish that by relying on a plausible counterfactual - the rejected applicants who were deemed eligible and managed to obtain data on long-term outcomes. The authors have put a lot of work in supporting their findings through various robustness checks. Furthermore, they not only estimate the direct effect of cash transfers on longevity but also examine the mechanisms through which these transfers might influence mortality. These short- and medium-term effects are in line with the results from more recent anti-poverty programs in both developed~\citep{almond2011inside, hoynes2016long, dahl2012impact, milligan2011child} and developing~\citep{barham2011healthier, barham2013living} countries.\\
\textbf{Main Assumptions:} Accepted and rejected applicants are the same on observable and unobservable characteristics and attrition is driven by mortality. They are tested for and supported through the robustness checks presented by the authors.   \\
\textbf{Effects sizes:} Possible underestimation of the effect of increase in income because of positive
externality from increased nutrition of the accepted to the rejected families. Hence, the boys of rejected applicants could benefit in terms of health. Possible overestimation of the increase in income from the MP transfers, through the channel of other programs, benefiting the rejected applicants. This will result in a smaller than the ``true'' gap between the two groups. Lastly, there is a possible underestimation of the effect on nutrition. Since the enlisted males in the WWII were of slightly better health, the estimates of the regression are likely to present a lower bound on the true effect of nutrition. \\
\textbf{External Validity:} There are significant differences in today's world compared to the beginning of the twentieth century. First, women have more labor market opportunities today and are a subject of more programs targeting female unemployment and family-friendly workplaces~\citep{lauber2016helping}. This may lead to different child-baring decisions, and furthermore, better combination between full- or part-time employment and having a child. Secondly, conditions without cash transfers might differ today. Not only support by the employers, but different support by the government in terms of child subsides. Third, families receiving cash assistance today could have changed their behavior. Lastly, the fact that the authors examine the effect of cash transfers on white males might be a big limitation in generalizing the results to females and black people.\\   
\textbf{Effectiveness:} Receivers might not maximize their childrens' well-being with the additional money or not use them efficiently. Furthermore, the funds might not be sufficient to drive long-term effects and only have short- and medium-run implications. 