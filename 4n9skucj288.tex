A. Aizer, S. Eli, J. Ferrie, and A. Lleras-Muney estimate the long-run
impact of unconditional cash transfers on children's longevity, income,
health and educational attainment. They use individual-level data of
the Mothers' Pension program and match it to census, WWII enlistment,
and DMF (Death Master File) records. The results suggest that compared
to children of rejected mothers, an accepted male child lived one year
longer, obtained one-third more years of schooling, was less likely to
be underweight, and had higher income in adulthood. The authors support their
findings through various robustness checks, verifying the reliability
of their assumptions. In the context of policy evaluation, the paper 
relates nicely to the first task defined by Heckman. It evaluates the impact of a historical intervention on the long-term outcomes and well-being of the treated. The authors identify the policy-relevant average treatment effect by choosing a suitable counterfactual. They go around the two main econometric problems (evaluation and selection problem) by using already eligible but rejected applicants. This way, their treatment status is not unobserved by the econometrician. Hence, by analyzing the differences between the two groups, the authors establish they are comparable and do not find a large statistically significant difference between them. However, a selection problem might occur if the participants can manipulate factors that determine their eligibility for the program participation and avoid rejection.